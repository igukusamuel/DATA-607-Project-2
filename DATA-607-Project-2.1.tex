\documentclass[]{article}
\usepackage{lmodern}
\usepackage{amssymb,amsmath}
\usepackage{ifxetex,ifluatex}
\usepackage{fixltx2e} % provides \textsubscript
\ifnum 0\ifxetex 1\fi\ifluatex 1\fi=0 % if pdftex
  \usepackage[T1]{fontenc}
  \usepackage[utf8]{inputenc}
\else % if luatex or xelatex
  \ifxetex
    \usepackage{mathspec}
  \else
    \usepackage{fontspec}
  \fi
  \defaultfontfeatures{Ligatures=TeX,Scale=MatchLowercase}
\fi
% use upquote if available, for straight quotes in verbatim environments
\IfFileExists{upquote.sty}{\usepackage{upquote}}{}
% use microtype if available
\IfFileExists{microtype.sty}{%
\usepackage{microtype}
\UseMicrotypeSet[protrusion]{basicmath} % disable protrusion for tt fonts
}{}
\usepackage[margin=1in]{geometry}
\usepackage{hyperref}
\hypersetup{unicode=true,
            pdftitle={DATA 607 Project 2},
            pdfauthor={Samuel I Kigamba},
            pdfborder={0 0 0},
            breaklinks=true}
\urlstyle{same}  % don't use monospace font for urls
\usepackage{color}
\usepackage{fancyvrb}
\newcommand{\VerbBar}{|}
\newcommand{\VERB}{\Verb[commandchars=\\\{\}]}
\DefineVerbatimEnvironment{Highlighting}{Verbatim}{commandchars=\\\{\}}
% Add ',fontsize=\small' for more characters per line
\usepackage{framed}
\definecolor{shadecolor}{RGB}{248,248,248}
\newenvironment{Shaded}{\begin{snugshade}}{\end{snugshade}}
\newcommand{\KeywordTok}[1]{\textcolor[rgb]{0.13,0.29,0.53}{\textbf{#1}}}
\newcommand{\DataTypeTok}[1]{\textcolor[rgb]{0.13,0.29,0.53}{#1}}
\newcommand{\DecValTok}[1]{\textcolor[rgb]{0.00,0.00,0.81}{#1}}
\newcommand{\BaseNTok}[1]{\textcolor[rgb]{0.00,0.00,0.81}{#1}}
\newcommand{\FloatTok}[1]{\textcolor[rgb]{0.00,0.00,0.81}{#1}}
\newcommand{\ConstantTok}[1]{\textcolor[rgb]{0.00,0.00,0.00}{#1}}
\newcommand{\CharTok}[1]{\textcolor[rgb]{0.31,0.60,0.02}{#1}}
\newcommand{\SpecialCharTok}[1]{\textcolor[rgb]{0.00,0.00,0.00}{#1}}
\newcommand{\StringTok}[1]{\textcolor[rgb]{0.31,0.60,0.02}{#1}}
\newcommand{\VerbatimStringTok}[1]{\textcolor[rgb]{0.31,0.60,0.02}{#1}}
\newcommand{\SpecialStringTok}[1]{\textcolor[rgb]{0.31,0.60,0.02}{#1}}
\newcommand{\ImportTok}[1]{#1}
\newcommand{\CommentTok}[1]{\textcolor[rgb]{0.56,0.35,0.01}{\textit{#1}}}
\newcommand{\DocumentationTok}[1]{\textcolor[rgb]{0.56,0.35,0.01}{\textbf{\textit{#1}}}}
\newcommand{\AnnotationTok}[1]{\textcolor[rgb]{0.56,0.35,0.01}{\textbf{\textit{#1}}}}
\newcommand{\CommentVarTok}[1]{\textcolor[rgb]{0.56,0.35,0.01}{\textbf{\textit{#1}}}}
\newcommand{\OtherTok}[1]{\textcolor[rgb]{0.56,0.35,0.01}{#1}}
\newcommand{\FunctionTok}[1]{\textcolor[rgb]{0.00,0.00,0.00}{#1}}
\newcommand{\VariableTok}[1]{\textcolor[rgb]{0.00,0.00,0.00}{#1}}
\newcommand{\ControlFlowTok}[1]{\textcolor[rgb]{0.13,0.29,0.53}{\textbf{#1}}}
\newcommand{\OperatorTok}[1]{\textcolor[rgb]{0.81,0.36,0.00}{\textbf{#1}}}
\newcommand{\BuiltInTok}[1]{#1}
\newcommand{\ExtensionTok}[1]{#1}
\newcommand{\PreprocessorTok}[1]{\textcolor[rgb]{0.56,0.35,0.01}{\textit{#1}}}
\newcommand{\AttributeTok}[1]{\textcolor[rgb]{0.77,0.63,0.00}{#1}}
\newcommand{\RegionMarkerTok}[1]{#1}
\newcommand{\InformationTok}[1]{\textcolor[rgb]{0.56,0.35,0.01}{\textbf{\textit{#1}}}}
\newcommand{\WarningTok}[1]{\textcolor[rgb]{0.56,0.35,0.01}{\textbf{\textit{#1}}}}
\newcommand{\AlertTok}[1]{\textcolor[rgb]{0.94,0.16,0.16}{#1}}
\newcommand{\ErrorTok}[1]{\textcolor[rgb]{0.64,0.00,0.00}{\textbf{#1}}}
\newcommand{\NormalTok}[1]{#1}
\usepackage{graphicx,grffile}
\makeatletter
\def\maxwidth{\ifdim\Gin@nat@width>\linewidth\linewidth\else\Gin@nat@width\fi}
\def\maxheight{\ifdim\Gin@nat@height>\textheight\textheight\else\Gin@nat@height\fi}
\makeatother
% Scale images if necessary, so that they will not overflow the page
% margins by default, and it is still possible to overwrite the defaults
% using explicit options in \includegraphics[width, height, ...]{}
\setkeys{Gin}{width=\maxwidth,height=\maxheight,keepaspectratio}
\IfFileExists{parskip.sty}{%
\usepackage{parskip}
}{% else
\setlength{\parindent}{0pt}
\setlength{\parskip}{6pt plus 2pt minus 1pt}
}
\setlength{\emergencystretch}{3em}  % prevent overfull lines
\providecommand{\tightlist}{%
  \setlength{\itemsep}{0pt}\setlength{\parskip}{0pt}}
\setcounter{secnumdepth}{0}
% Redefines (sub)paragraphs to behave more like sections
\ifx\paragraph\undefined\else
\let\oldparagraph\paragraph
\renewcommand{\paragraph}[1]{\oldparagraph{#1}\mbox{}}
\fi
\ifx\subparagraph\undefined\else
\let\oldsubparagraph\subparagraph
\renewcommand{\subparagraph}[1]{\oldsubparagraph{#1}\mbox{}}
\fi

%%% Use protect on footnotes to avoid problems with footnotes in titles
\let\rmarkdownfootnote\footnote%
\def\footnote{\protect\rmarkdownfootnote}

%%% Change title format to be more compact
\usepackage{titling}

% Create subtitle command for use in maketitle
\providecommand{\subtitle}[1]{
  \posttitle{
    \begin{center}\large#1\end{center}
    }
}

\setlength{\droptitle}{-2em}

  \title{DATA 607 Project 2}
    \pretitle{\vspace{\droptitle}\centering\huge}
  \posttitle{\par}
    \author{Samuel I Kigamba}
    \preauthor{\centering\large\emph}
  \postauthor{\par}
      \predate{\centering\large\emph}
  \postdate{\par}
    \date{October 06, 2019}


\begin{document}
\maketitle

{
\setcounter{tocdepth}{2}
\tableofcontents
}
\section{--------------------------------------------------------------------------------}\label{section}

\section{\texorpdfstring{\clearpage}{}}\label{section-1}

DATA 607 Project 2.

The goal of this assignment is to give you practice in preparing
different datasets for downstream analysis work. Your task is to: (1)
Choose any three of the ``wide'' datasets identified in the Week 5
Discussion items. (You may use your own dataset; please don't use my
Sample Post dataset, since that was used in your Week 6 assignment!) For
each of the three chosen datasets: ??? Create a .CSV file (or
optionally, a MySQL database!) that includes all of the information
included in the dataset. You're encouraged to use a ``wide'' structure
similar to how the information appears in the discussion item, so that
you can practice tidying and transformations as described below. ???
Read the information from your .CSV file into R, and use tidyr and dplyr
as needed to tidy and transform your data. {[}Most of your grade will be
based on this step!{]} ??? Perform the analysis requested in the
discussion item. ??? Your code should be in an R Markdown file, posted
to rpubs.com, and should include narrative descriptions of your data
cleanup work, analysis, and conclusions. (2) Please include in your
homework submission, for each of the three chosen datasets: ??? The URL
to the .Rmd file in your GitHub repository, and ??? The URL for your
rpubs.com web page.

set working directory and Install all the relevant packages and load
their respective libraries into R.

\section{Male migrants}\label{male-migrants}

\subsection{Load the following
libraries}\label{load-the-following-libraries}

\paragraph{library(stringr)}\label{librarystringr}

\paragraph{library(tidyr)}\label{librarytidyr}

\paragraph{library(dplyr)}\label{librarydplyr}

\paragraph{library(tidyverse)}\label{librarytidyverse}

\paragraph{library(tibble)}\label{librarytibble}

\paragraph{library(caret)}\label{librarycaret}

\paragraph{library(readr)}\label{libraryreadr}

\subsection{Upload the data into
Github}\label{upload-the-data-into-github}

This will ensure that everyone with access to the github repository can
easily audit or retest the data. This ensures ease of accessibility and
testing by a wide audience. Follow this link to see uploaded Male
migrants .csv file
(\url{https://raw.githubusercontent.com/igukusamuel/DATA-607-Project-2/master/UN_MigrantStockBySexByDestination_2019.csv})

\begin{Shaded}
\begin{Highlighting}[]
\NormalTok{migrants <-}\StringTok{ }\KeywordTok{read_csv}\NormalTok{(}\StringTok{"https://raw.githubusercontent.com/igukusamuel/DATA-607-Project-2/master/UN_MigrantStockBySexByDestination_2019.csv"}\NormalTok{)}
\KeywordTok{head}\NormalTok{(migrants)}
\end{Highlighting}
\end{Shaded}

\begin{verbatim}
## # A tibble: 6 x 26
##   X1    X2    X3    X4    X5      X6    X7    X8    X9    X10   X11   X12  
##   <chr> <chr> <chr> <chr> <chr>   <chr> <chr> <chr> <chr> <chr> <chr> <chr>
## 1 <NA>  <NA>  <NA>  <NA>  <NA>    <NA>  <NA>  <NA>  <NA>  <NA>  <NA>  <NA> 
## 2 <NA>  <NA>  <NA>  <NA>  <NA>    <NA>  <NA>  <NA>  <NA>  <NA>  <NA>  <NA> 
## 3 <NA>  <NA>  <NA>  <NA>  <NA>    <NA>  <NA>  <NA>  <NA>  <NA>  <NA>  <NA> 
## 4 <NA>  <NA>  <NA>  <NA>  United~ <NA>  <NA>  <NA>  <NA>  <NA>  <NA>  <NA> 
## 5 <NA>  <NA>  <NA>  <NA>  Popula~ <NA>  <NA>  <NA>  <NA>  <NA>  <NA>  <NA> 
## 6 <NA>  <NA>  <NA>  <NA>  Depart~ <NA>  <NA>  <NA>  <NA>  <NA>  <NA>  <NA> 
## # ... with 14 more variables: X13 <chr>, X14 <chr>, X15 <chr>, X16 <chr>,
## #   X17 <chr>, X18 <chr>, X19 <chr>, X20 <chr>, X21 <chr>, X22 <chr>,
## #   X23 <chr>, X24 <chr>, X25 <chr>, X26 <chr>
\end{verbatim}

\begin{Shaded}
\begin{Highlighting}[]
\CommentTok{#view(head(male_migrants, 20)) # vIew data frame structure and see how many rows to skip.}
\end{Highlighting}
\end{Shaded}

\subsection{Skip first 15 rows}\label{skip-first-15-rows}

As part of data cleanup, skip the first 15 rows that include source
information not relevant to out analysis.

\begin{Shaded}
\begin{Highlighting}[]
\NormalTok{migrants <-}\StringTok{ }\KeywordTok{read_csv}\NormalTok{(}\StringTok{"https://raw.githubusercontent.com/igukusamuel/DATA-607-Project-2/master/UN_MigrantStockBySexByDestination_2019.csv"}\NormalTok{, }\DataTypeTok{skip =} \DecValTok{15}\NormalTok{)}

\KeywordTok{head}\NormalTok{(migrants) }\CommentTok{#Print out first few rows to confirm that the data have been loaded correctly.}
\end{Highlighting}
\end{Shaded}

\begin{verbatim}
## # A tibble: 6 x 26
##      X1 X2    X3       X4 X5    `1990` `1995` `2000` `2005` `2010` `2015`
##   <dbl> <chr> <chr> <dbl> <chr> <chr>  <chr>  <chr>  <chr>  <chr>  <chr> 
## 1     1 WORLD <NA>    900 <NA>  153,0~ 161,3~ 173,5~ 191,6~ 220,7~ 248,8~
## 2     2 UN d~ <NA>     NA <NA>  ..     ..     ..     ..     ..     ..    
## 3     3 More~ b       901 <NA>  82,76~ 92,93~ 103,9~ 116,6~ 130,6~ 140,6~
## 4     4 Less~ c       902 <NA>  70,24~ 68,38~ 69,62~ 74,92~ 90,16~ 108,2~
## 5     5 Leas~ d       941 <NA>  11,06~ 11,68~ 10,06~ 9,833~ 10,43~ 13,63~
## 6     6 Less~ <NA>    934 <NA>  59,18~ 56,70~ 59,56~ 65,09~ 79,73~ 94,58~
## # ... with 15 more variables: `2019` <chr>, `1990_1` <chr>,
## #   `1995_1` <chr>, `2000_1` <chr>, `2005_1` <chr>, `2010_1` <chr>,
## #   `2015_1` <chr>, `2019_1` <chr>, `1990_2` <chr>, `1995_2` <chr>,
## #   `2000_2` <chr>, `2005_2` <chr>, `2010_2` <chr>, `2015_2` <chr>,
## #   `2019_2` <chr>
\end{verbatim}

\subsection{Filter for N/As in column
X6}\label{filter-for-nas-in-column-x6}

Careful review of the data shows that column named X5 only includes data
for rows related to countries and N/A's for rows relating to regions and
regional totals. Thus filtering out all N/As in column X5 will leave us
with country data only, which is the basis of out analysis. We first
view all the N/As under column X5 to confirm none of them relate to
country information.

\begin{Shaded}
\begin{Highlighting}[]
\NormalTok{colX5 <-}\StringTok{ }\KeywordTok{filter}\NormalTok{(migrants, }\KeywordTok{is.na}\NormalTok{(X5))}

\NormalTok{x <-}\StringTok{ }\KeywordTok{length}\NormalTok{(colX5)}
\NormalTok{x}
\end{Highlighting}
\end{Shaded}

\begin{verbatim}
## [1] 26
\end{verbatim}

\begin{Shaded}
\begin{Highlighting}[]
\KeywordTok{head}\NormalTok{(colX5)}
\end{Highlighting}
\end{Shaded}

\begin{verbatim}
## # A tibble: 6 x 26
##      X1 X2    X3       X4 X5    `1990` `1995` `2000` `2005` `2010` `2015`
##   <dbl> <chr> <chr> <dbl> <chr> <chr>  <chr>  <chr>  <chr>  <chr>  <chr> 
## 1     1 WORLD <NA>    900 <NA>  153,0~ 161,3~ 173,5~ 191,6~ 220,7~ 248,8~
## 2     2 UN d~ <NA>     NA <NA>  ..     ..     ..     ..     ..     ..    
## 3     3 More~ b       901 <NA>  82,76~ 92,93~ 103,9~ 116,6~ 130,6~ 140,6~
## 4     4 Less~ c       902 <NA>  70,24~ 68,38~ 69,62~ 74,92~ 90,16~ 108,2~
## 5     5 Leas~ d       941 <NA>  11,06~ 11,68~ 10,06~ 9,833~ 10,43~ 13,63~
## 6     6 Less~ <NA>    934 <NA>  59,18~ 56,70~ 59,56~ 65,09~ 79,73~ 94,58~
## # ... with 15 more variables: `2019` <chr>, `1990_1` <chr>,
## #   `1995_1` <chr>, `2000_1` <chr>, `2005_1` <chr>, `2010_1` <chr>,
## #   `2015_1` <chr>, `2019_1` <chr>, `1990_2` <chr>, `1995_2` <chr>,
## #   `2000_2` <chr>, `2005_2` <chr>, `2010_2` <chr>, `2015_2` <chr>,
## #   `2019_2` <chr>
\end{verbatim}

\subsection{Exclude N/As in column X5}\label{exclude-nas-in-column-x5}

We then exclude all N/A's in column X6 and print out the first 6 rows
using the head() function.

\begin{Shaded}
\begin{Highlighting}[]
\NormalTok{migrants_by_country <-}\StringTok{ }\KeywordTok{filter}\NormalTok{(migrants, }\OperatorTok{!}\KeywordTok{is.na}\NormalTok{(X5))}

\KeywordTok{head}\NormalTok{(migrants_by_country)}
\end{Highlighting}
\end{Shaded}

\begin{verbatim}
## # A tibble: 6 x 26
##      X1 X2    X3       X4 X5    `1990` `1995` `2000` `2005` `2010` `2015`
##   <dbl> <chr> <chr> <dbl> <chr> <chr>  <chr>  <chr>  <chr>  <chr>  <chr> 
## 1    24 Buru~ <NA>    108 B R   333,1~ 254,8~ 125,6~ 172,8~ 235,2~ 289,8~
## 2    25 Como~ <NA>    174 B     14,079 13,939 13,799 13,209 12,618 12,555
## 3    26 Djib~ <NA>    262 B R   122,2~ 99,774 100,5~ 92,091 101,5~ 112,3~
## 4    27 Erit~ <NA>    232 I     11,848 12,400 12,952 14,314 15,676 15,941
## 5    28 Ethi~ <NA>    231 B R   1,155~ 806,9~ 611,3~ 514,2~ 568,7~ 1,161~
## 6    29 Kenya <NA>    404 B R   298,0~ 618,7~ 707,8~ 773,3~ 954,9~ 1,126~
## # ... with 15 more variables: `2019` <chr>, `1990_1` <chr>,
## #   `1995_1` <chr>, `2000_1` <chr>, `2005_1` <chr>, `2010_1` <chr>,
## #   `2015_1` <chr>, `2019_1` <chr>, `1990_2` <chr>, `1995_2` <chr>,
## #   `2000_2` <chr>, `2005_2` <chr>, `2010_2` <chr>, `2015_2` <chr>,
## #   `2019_2` <chr>
\end{verbatim}

\subsection{Rename column X2}\label{rename-column-x2}

From the above print out, there is need to rename column X2
dest\_country.

\begin{Shaded}
\begin{Highlighting}[]
\NormalTok{migrants_by_country <-}\StringTok{ }\NormalTok{migrants_by_country }\OperatorTok\StringTok{ }
\StringTok{        }\KeywordTok{rename}\NormalTok{(}
                \DataTypeTok{dest_country =}\NormalTok{ X2}
\NormalTok{        )}
\KeywordTok{head}\NormalTok{(migrants_by_country)}
\end{Highlighting}
\end{Shaded}

\begin{verbatim}
## # A tibble: 6 x 26
##      X1 dest_country X3       X4 X5    `1990` `1995` `2000` `2005` `2010`
##   <dbl> <chr>        <chr> <dbl> <chr> <chr>  <chr>  <chr>  <chr>  <chr> 
## 1    24 Burundi      <NA>    108 B R   333,1~ 254,8~ 125,6~ 172,8~ 235,2~
## 2    25 Comoros      <NA>    174 B     14,079 13,939 13,799 13,209 12,618
## 3    26 Djibouti     <NA>    262 B R   122,2~ 99,774 100,5~ 92,091 101,5~
## 4    27 Eritrea      <NA>    232 I     11,848 12,400 12,952 14,314 15,676
## 5    28 Ethiopia     <NA>    231 B R   1,155~ 806,9~ 611,3~ 514,2~ 568,7~
## 6    29 Kenya        <NA>    404 B R   298,0~ 618,7~ 707,8~ 773,3~ 954,9~
## # ... with 16 more variables: `2015` <chr>, `2019` <chr>, `1990_1` <chr>,
## #   `1995_1` <chr>, `2000_1` <chr>, `2005_1` <chr>, `2010_1` <chr>,
## #   `2015_1` <chr>, `2019_1` <chr>, `1990_2` <chr>, `1995_2` <chr>,
## #   `2000_2` <chr>, `2005_2` <chr>, `2010_2` <chr>, `2015_2` <chr>,
## #   `2019_2` <chr>
\end{verbatim}

\subsection{View all columns}\label{view-all-columns}

The above printout shows a number of irrelevant columns that are not
necessary for our analysis. Lets print out the entire column names and
delete the unnecessary ones to have a cleaner data set.

\begin{Shaded}
\begin{Highlighting}[]
\NormalTok{column_names <-}\StringTok{ }\KeywordTok{colnames}\NormalTok{(migrants_by_country)}
\CommentTok{#column_names # umcomment to view entire list of column names}
\KeywordTok{head}\NormalTok{(column_names)}
\end{Highlighting}
\end{Shaded}

\begin{verbatim}
## [1] "X1"           "dest_country" "X3"           "X4"          
## [5] "X5"           "1990"
\end{verbatim}

\subsection{Exclude irrelevant
columns}\label{exclude-irrelevant-columns}

The above print out reveals that we do not need all column names that
start with ``X''. We delete these columns using the srtarts\_with
function.

\begin{Shaded}
\begin{Highlighting}[]
\NormalTok{migrants_by_country <-}\StringTok{ }\NormalTok{migrants_by_country }\OperatorTok\StringTok{ }
\StringTok{        }\KeywordTok{select}\NormalTok{(}\OperatorTok{-}\KeywordTok{starts_with}\NormalTok{(}\StringTok{"X"}\NormalTok{))}


\NormalTok{migrants_by_country <-}\StringTok{ }\NormalTok{migrants_by_country }\OperatorTok\StringTok{ }
\StringTok{        }\KeywordTok{select}\NormalTok{(}\OperatorTok{-}\KeywordTok{c}\NormalTok{(}\DecValTok{2}\OperatorTok{:}\DecValTok{8}\NormalTok{))}

\NormalTok{migrants_by_country}
\end{Highlighting}
\end{Shaded}

\begin{verbatim}
## # A tibble: 232 x 15
##    dest_country `1990_1` `1995_1` `2000_1` `2005_1` `2010_1` `2015_1`
##    <chr>        <chr>    <chr>    <chr>    <chr>    <chr>    <chr>   
##  1 Burundi      163,267  124,165  61,094   84,805   115,823  142,790 
##  2 Comoros      6,717    6,614    6,511    6,286    6,060    6,071   
##  3 Djibouti     64,242   52,476   52,920   51,315   53,295   59,081  
##  4 Eritrea      6,228    6,542    6,856    7,729    8,603    8,833   
##  5 Ethiopia     607,284  424,117  322,219  269,725  298,069  591,409 
##  6 Kenya        161,259  322,189  352,933  400,364  473,093  562,909 
##  7 Madagascar   13,348   11,901   13,276   14,744   16,410   18,270  
##  8 Malawi       546,520  116,198  111,530  105,931  103,869  110,893 
##  9 Mauritius    1,763    3,228    5,705    8,943    13,188   15,832  
## 10 Mayotte      8,780    14,679   23,546   31,364   34,500   34,235  
## # ... with 222 more rows, and 8 more variables: `2019_1` <chr>,
## #   `1990_2` <chr>, `1995_2` <chr>, `2000_2` <chr>, `2005_2` <chr>,
## #   `2010_2` <chr>, `2015_2` <chr>, `2019_2` <chr>
\end{verbatim}

\subsection{View dimentions of resulting data
frame}\label{view-dimentions-of-resulting-data-frame}

We use dim() function to have an idea of how many rows and columns we
have for our analysis.

\begin{Shaded}
\begin{Highlighting}[]
\KeywordTok{dim}\NormalTok{(migrants_by_country)}
\end{Highlighting}
\end{Shaded}

\begin{verbatim}
## [1] 232  15
\end{verbatim}

\subsection{Confrim column names.}\label{confrim-column-names.}

This is what we need for our analysis.

\begin{Shaded}
\begin{Highlighting}[]
\NormalTok{column_names_clean <-}\StringTok{ }\KeywordTok{colnames}\NormalTok{(migrants_by_country)}
\CommentTok{#column_names_clean # uncomment to view entire list of cleaned up column names}
\KeywordTok{head}\NormalTok{(column_names_clean)}
\end{Highlighting}
\end{Shaded}

\begin{verbatim}
## [1] "dest_country" "1990_1"       "1995_1"       "2000_1"      
## [5] "2005_1"       "2010_1"
\end{verbatim}

\subsection{View number of columns}\label{view-number-of-columns}

Get the length of the column names to be used in the next line of code.

\begin{Shaded}
\begin{Highlighting}[]
\NormalTok{y <-}\StringTok{ }\KeywordTok{length}\NormalTok{(}\KeywordTok{colnames}\NormalTok{(migrants_by_country))}

\NormalTok{y }
\end{Highlighting}
\end{Shaded}

\begin{verbatim}
## [1] 15
\end{verbatim}

\subsection{clean up data}\label{clean-up-data}

Let us use gather() function to gather all columns with years into a
single columns and exclude any and all N/As to obtain clean data. Spread
the resulting data by year column and rename ``1'' as male and ``2'' as
female.

\begin{Shaded}
\begin{Highlighting}[]
\NormalTok{no_of_migrants_per_country <-}\StringTok{ }\KeywordTok{mutate}\NormalTok{(}\KeywordTok{gather}\NormalTok{(migrants_by_country, }\StringTok{"year"}\NormalTok{, }\StringTok{"no_of_migrants"}\NormalTok{, }\DecValTok{2}\OperatorTok{:}\NormalTok{y, }\DataTypeTok{na.rm =} \OtherTok{TRUE}\NormalTok{))}

\KeywordTok{head}\NormalTok{(no_of_migrants_per_country)}
\end{Highlighting}
\end{Shaded}

\begin{verbatim}
## # A tibble: 6 x 3
##   dest_country year   no_of_migrants
##   <chr>        <chr>  <chr>         
## 1 Burundi      1990_1 163,267       
## 2 Comoros      1990_1 6,717         
## 3 Djibouti     1990_1 64,242        
## 4 Eritrea      1990_1 6,228         
## 5 Ethiopia     1990_1 607,284       
## 6 Kenya        1990_1 161,259
\end{verbatim}

\begin{Shaded}
\begin{Highlighting}[]
\NormalTok{no_of_migrants_per_country <-}\StringTok{ }\NormalTok{no_of_migrants_per_country }\OperatorTok
\StringTok{        }\KeywordTok{separate}\NormalTok{(year, }\KeywordTok{c}\NormalTok{(}\StringTok{"year"}\NormalTok{, }\StringTok{"sex"}\NormalTok{), }\DataTypeTok{sep =} \StringTok{"_"}\NormalTok{)}

\NormalTok{no_of_migrants_per_country}
\end{Highlighting}
\end{Shaded}

\begin{verbatim}
## # A tibble: 3,248 x 4
##    dest_country year  sex   no_of_migrants
##    <chr>        <chr> <chr> <chr>         
##  1 Burundi      1990  1     163,267       
##  2 Comoros      1990  1     6,717         
##  3 Djibouti     1990  1     64,242        
##  4 Eritrea      1990  1     6,228         
##  5 Ethiopia     1990  1     607,284       
##  6 Kenya        1990  1     161,259       
##  7 Madagascar   1990  1     13,348        
##  8 Malawi       1990  1     546,520       
##  9 Mauritius    1990  1     1,763         
## 10 Mayotte      1990  1     8,780         
## # ... with 3,238 more rows
\end{verbatim}

Convert the years column to number format

\begin{Shaded}
\begin{Highlighting}[]
\NormalTok{no_of_migrants_per_country}\OperatorTok{$}\NormalTok{year <-}\StringTok{ }\KeywordTok{parse_number}\NormalTok{(no_of_migrants_per_country}\OperatorTok{$}\NormalTok{year)}

\NormalTok{no_of_migrants_per_country}
\end{Highlighting}
\end{Shaded}

\begin{verbatim}
## # A tibble: 3,248 x 4
##    dest_country  year sex   no_of_migrants
##    <chr>        <dbl> <chr> <chr>         
##  1 Burundi       1990 1     163,267       
##  2 Comoros       1990 1     6,717         
##  3 Djibouti      1990 1     64,242        
##  4 Eritrea       1990 1     6,228         
##  5 Ethiopia      1990 1     607,284       
##  6 Kenya         1990 1     161,259       
##  7 Madagascar    1990 1     13,348        
##  8 Malawi        1990 1     546,520       
##  9 Mauritius     1990 1     1,763         
## 10 Mayotte       1990 1     8,780         
## # ... with 3,238 more rows
\end{verbatim}

\begin{Shaded}
\begin{Highlighting}[]
\NormalTok{no_of_migrants_per_country <-}\StringTok{ }\NormalTok{no_of_migrants_per_country }\OperatorTok
\StringTok{        }\KeywordTok{spread}\NormalTok{(sex, no_of_migrants)}
        

\KeywordTok{names}\NormalTok{(no_of_migrants_per_country)}
\end{Highlighting}
\end{Shaded}

\begin{verbatim}
## [1] "dest_country" "year"         "1"            "2"
\end{verbatim}

\begin{Shaded}
\begin{Highlighting}[]
\NormalTok{no_of_migrants_per_country <-}\StringTok{ }\NormalTok{no_of_migrants_per_country }\OperatorTok\StringTok{ }
\StringTok{        }\KeywordTok{rename}\NormalTok{(}
                \DataTypeTok{male =} \StringTok{"1"}\NormalTok{,}
                \DataTypeTok{female =} \StringTok{"2"}
\NormalTok{        )}
\KeywordTok{head}\NormalTok{(no_of_migrants_per_country)}
\end{Highlighting}
\end{Shaded}

\begin{verbatim}
## # A tibble: 6 x 4
##   dest_country  year male    female 
##   <chr>        <dbl> <chr>   <chr>  
## 1 Afghanistan   1990 32,558  25,128 
## 2 Afghanistan   1995 39,105  32,417 
## 3 Afghanistan   2000 42,848  33,069 
## 4 Afghanistan   2005 49,274  38,026 
## 5 Afghanistan   2010 57,709  44,537 
## 6 Afghanistan   2015 248,212 241,537
\end{verbatim}

\subsection{Conversion of chr to dbl}\label{conversion-of-chr-to-dbl}

convert the no\_of\_migrants data column from characters to doubles for
statistical analysis. This we will do using the parse\_number()
function. Print out using head() function the first 6 rows and confirm
this conversion.

\begin{Shaded}
\begin{Highlighting}[]
\NormalTok{no_of_migrants_per_country}\OperatorTok{$}\NormalTok{male <-}\StringTok{ }\KeywordTok{parse_number}\NormalTok{(no_of_migrants_per_country}\OperatorTok{$}\NormalTok{male)}
\NormalTok{no_of_migrants_per_country}\OperatorTok{$}\NormalTok{female <-}\StringTok{ }\KeywordTok{parse_number}\NormalTok{(no_of_migrants_per_country}\OperatorTok{$}\NormalTok{female)}

\NormalTok{clean_migrants_data <-}\StringTok{ }\NormalTok{no_of_migrants_per_country}

\KeywordTok{head}\NormalTok{(clean_migrants_data)}
\end{Highlighting}
\end{Shaded}

\begin{verbatim}
## # A tibble: 6 x 4
##   dest_country  year   male female
##   <chr>        <dbl>  <dbl>  <dbl>
## 1 Afghanistan   1990  32558  25128
## 2 Afghanistan   1995  39105  32417
## 3 Afghanistan   2000  42848  33069
## 4 Afghanistan   2005  49274  38026
## 5 Afghanistan   2010  57709  44537
## 6 Afghanistan   2015 248212 241537
\end{verbatim}


\end{document}
